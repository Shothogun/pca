\documentclass{beamer}
\usetheme{Frankfurt}

\usepackage{algpseudocodex}

\usepackage{pgfpages}
\usepackage{awesomebox}
\usepackage{amsmath}

\setbeamertemplate{note page}{\pagecolor{yellow!5}\insertnote}\usepackage{palatino}

\AtBeginSection[]{
  \begin{frame}
  \vfill
  \centering
  \begin{beamercolorbox}[sep=8pt,center,shadow=true,rounded=true]{title}
    \usebeamerfont{title}\insertsectionhead\par%
  \end{beamercolorbox}
  \vfill
  \end{frame}
}

%\setbeameroption{show notes on second screen=right} % Both

\title{Design and Analysis of Algorithms}

\author{Rodrigo Bonif\'{a}cio}

\begin{document}
\begin{frame}
\titlepage
\end{frame}

\section{Introduction}

\begin{frame}

  \begin{block}{Introduction to Algorithms (Cormen et al.)}
  \begin{itemize}
    \item An algorithm is any well-defined computational procedure (sequence of
    steps) that takes some set of values as input and produces some set of values
    as output. \pause

  \item An algorithm is a tool for solving a well-defined computational problem. The
    statement of the problem specifies the input/output relationship. The
    algorithm describes a specific computational procedure for achieving the
    input/output relationship.
  \end{itemize}
  \end{block}

  
  \note[item]{The input must satisfy whatever constraints imposed in the problem statement. Example:
  the binary search algorithm requires as input an ordered sequence.}
  
\end{frame}

\begin{frame}

  \begin{block}{The Art of Computer Programming - Volume I - (D. Knuth)}
    \begin{itemize}
      \item The modern meaning of algorithm is similar to that of
      a method, procedure, routine, and the like. \pause An algorithm must
      additionally include five important features\pause:

      \begin{enumerate}
       \item Finiteness
       \item Definiteness
       \item Input
       \item Output
       \item Effectiveness \pause(in terms of both time and space, for instance)  
      \end{enumerate}
   \end{itemize}   
  \end{block}  
\end{frame}

\begin{frame}{Different domains are fertile of computational problems}

  \begin{itemize}
   \item Biology (e.g., DNA sequencing)
   \item Electronic commerce (e.g., ML, cryptography)
   \item Manufacturing (optimization)
   \item Compilers (parsers, data-flow analysis, \ldots)
  \end{itemize}
  
\end{frame}

\begin{frame}{Goals of this course}
  \begin{itemize}
   \item Data Structures (graphs)
   \item Techniques for {\color{blue}algorithm design and analysis}
     \begin{itemize}
       \item divide-and-conquer
       \item dynamic programming, greedy algorithms
       \item chaotic, fixing-point algorithms, \ldots  
     \end{itemize}
   \item Hard problems (aka {\color{blue}complexity} theory) 
  \end{itemize}
\end{frame}


\begin{frame}{Example: Sorting problem}

  \begin{description}
   \item[Input] A sequence of $n$ numbers $\langle a_1, a_2, \ldots, a_n \rangle$
   \item[Output] A permutation (reordering) $\langle a'_1, a'_2, \ldots, a'_n \rangle$ of the input
     sequence, such that $a'_1 \leq a'_2 \leq \ldots \leq a'_n$
  \end{description}

  \pause
  
  \begin{block}{Problem Instance}
    Given the input sequence $\langle 31, 41, 59, 26, 41, 58\rangle$,
    a sorting algorithm returns as output the sequence: \pause
    $\langle 26, 31, 41, 41, 58, 59 \rangle$
  \end{block}


\end{frame}

\begin{frame}
  Sorting is a fundamental operation in Computer Science. \pause As such,
  we have a large number of good algorithms at our disposal. \pause

  \begin{block}{Fundamental questions}
   \begin{itemize}
   \item what is the best sorting algorithm? (analysis)
   \item is this algorithm correct? (correctness proof)
   \end{itemize}  
  \end{block}
\end{frame}


\begin{frame}{Insertion Sort}
  
  \begin{itemize}
  \item An efficient algorithm for sorting an array {\bf A} with a
    {\color{blue}small number of elements}. \pause

  \item The algorithm sorts the input numbers in place---i.e., rearranging the
    numbers within the input array {\bf A} (emphasis on the procedural style). \pause 

  \item For convenience, we will assume that the indexes of an array of size $n$ are numbered
    from $1 .. n$ in the pseudocode of the algorithms. \pause The implementation assumes
    the arrays (or lists) start in the index $0$.   
  \end{itemize}
\end{frame}

\begin{frame}
  \begin{block}{Pseudocode for Insertion Sort}
  \begin{algorithmic}
    \Procedure{InsertionSort}{A}
       \For{$j = 2, \dots, A.length$} 
         \State $key \gets A[j]$
         \State $i \gets j-1$
         \While{$i > 0\ and\ A[i] > key$}
            \State $A[i+1] \gets A[i]$
            \State $i \gets i - 1$        
         \EndWhile
         \State $A[i+1] = key$  
       \EndFor
    \EndProcedure
  \end{algorithmic}  
  \end{block}

  \pause 

  \begin{block}{Analysis}
    \begin{itemize}
      \item expected (worst-case order of growth) running time
      \item correctness proof (loop invariant) 
    \end{itemize}

    \note[item]{The running time of an algorithm on a particular input is the number
      of (machine-independent) primitive operations executed.}

    \note[item]{We should consider three aspects: initialization, maintenance, and
    termination. This follows the proof-by-induction method.}
  \end{block}

\end{frame}

\begin{frame}{Merge Sort}

    \begin{itemize}
      \item a ``divide-and-conquer'' approach for sorting.
      \item worst-case running time is much less than that of insertion sort\pause:
        \begin{itemize}
         \item insertion sort: $\mathcal{O}(n^2)$ 
         \item merge sort: $\mathcal{O}(n\log{}n)$ 
        \end{itemize}
      \item Essence of the divide-and-conquer approach: many algorithm are {\color{blue}recursive} in nature. \pause
        We can decompose the problems in three steps:
        \begin{itemize}
         \item Divide
         \item Conquer
         \item Combine  
        \end{itemize}
        \pause
      \item For the Merge Sort algorithm, the {\color{blue}Combine} step is the interest one. \pause Let's discuss
        two variants of the merge sort. 
    \end{itemize}
\end{frame}

\begin{frame}
\begin{block}{(V1) The merge function returns a new, sorted array}
\begin{scriptsize}
    \begin{algorithmic}
      \Procedure{Merge}{A,B}
%       \comment{A and B are sorted arrays}
       \State $Size \gets A.length + B.length$ 
       \State $i, j, k \gets 1$  
       \State $C \gets Array[1 .. Size]$ 

%       \comment{Iterate over the elements of A and B, and copy them in sorted order}
       \While{$i \leq A.length\ and\ j \leq B.length$}
         \If{$A[i] > B[j]$}
           \State $C[k] \gets A[i]$
           \State $i \gets i + 1$
         \Else
           \State $C[k] \gets B[j]$
           \State $j \gets j + 1$
         \EndIf
         \State $k \gets k + 1$  
       \EndWhile
         
%% %       \comment{Copy the remaining elements of A}  
       \For{$l = i, \ldots, A.length$}
         \State $C[k] \gets A[l]$
         \State $k \gets k + 1$
       \EndFor

%% %       \comment{Copy the remaining elements of B} 
       \For{$l = j, \ldots, B.length$}
        \State $C[k] \gets B[l]$
        \State $k \gets k + 1$
       \EndFor 
       \State {\bf return} $C$
    \EndProcedure
    \end{algorithmic}
\end{scriptsize}
\end{block}
\end{frame}

\begin{frame}{Sorting with Merge Sort}

  \begin{itemize}
    \item The sort procedure is trivial. 
  \end{itemize}

  \pause

  \begin{block}{Top-level procedure of merge sort}
    \begin{scriptsize}
    \begin{algorithmic}
      \Procedure{MergeSort}{A}
        \If{$A.length \leq 1$}
         \State {\bf return} $A$
        \EndIf

        \State $Mid \gets A.length / 2$
        \State $Left  \gets MergeSort(A[1 \ldots Mid])$
        \State $Right \gets MergeSort(A[Mid \ldots A.length])$

        \State {\bf return} $Merge(Left, Right)$ 
      \EndProcedure
    \end{algorithmic}
    \end{scriptsize}
  \end{block}   
\end{frame}

\begin{frame}
\begin{block}{(V2): More ``procedural'' approach}
\begin{scriptsize}
    \begin{algorithmic}
      \Procedure{Merge}{A, p, q, r}
       \State $n1 \gets q - p + 1$
       \State $n2 \gets r - q$  
       \State $L \gets Array[1 .. n1+1]$
       \State $R \gets Array[1 .. n2+1]$

       \For{$i = 1, \ldots, n1$}
         \State{$L[i] \gets A[p+i-1]$}
       \EndFor
           
       \For{$j = 1, \ldots, n2$}
         \State{$R[j] \gets A[q+j]$}
       \EndFor

       \State{$L[n1+1] \gets \infty$}
       \State{$R[n2+1] \gets \infty$}
       
       \State{$i \gets 1$}
       \State{$j \gets 1$}

       \For{$k = p, \ldots, r$}

         \If{$L[i] \leq R[j]$}
           \State{$A[k] \gets L[i]$}
           \State{$i \gets i + 1$}
         \Else
           \State{$A[k] \gets R[j]$}
           \State{$j \gets j + 1$}
         \EndIf
       \EndFor
    \EndProcedure
    \end{algorithmic}
\end{scriptsize}
\end{block}
\end{frame}

\begin{frame}{Divide-and-Conquer}
  \begin{block}{Remember}
  \begin{itemize}  
  \item We decompose the problems in three steps:
        \begin{itemize}
        \item Divide
        \item Conquer
        \item Combine  
        \end{itemize}
      \item We can solve many problems using this approach.
  \end{itemize}      
  \end{block}
\end{frame}

\begin{frame}{The {\bf Maximum Subarray} problem}
  \begin{itemize}
    \item Goal: Find the nonempty, contiguous subarray of an
      array {\bf A}, whose values have the largest sum.
  \end{itemize} \pause
  
  \begin{block}{Example: consider the Array}
    \vskip+1em
    \begin{center}   
      \begin{scriptsize}$A = \langle 13, -3, -25, 20, -3, -16, -23, 18, 20, -7, 12, -5, -22, 15, -4, 7 \rangle$\end{scriptsize}
    \end{center}

    \begin{itemize}
      \item $maxSubarray(A) = (8, 11, 43)$ 
    \end{itemize}
  \end{block}  

\end{frame}

\begin{frame}
  \begin{itemize}
    \item How to solve the Maximum Subarray problem using the {\color{blue}divide-and-conquer} method? \pause
  \end{itemize}


  \begin{block}{General idea}
    \begin{small}  
    \begin{algorithmic}
    \Procedure{MaxSubArray}{A}
       \If{$A.length = 1$}
         \State {\bf return} $(1, 1, A[1])$ \Comment{base case} 
       \EndIf

       \State
       
       \State $Mid \gets \lfloor A.length / 2 \rfloor$

       \State 
       \State $Left \gets MaxSubArray(A[1 \ldots Mid])$
       \State $Right \gets MaxSubArray(A[(Mid + 1) \ldots A.length])$
       \State $Crossing \gets MaxCrossing(A)$

       \State 
       \State {\bf return} $Max(Left, Right, Crossing)$
      \EndProcedure 
    \end{algorithmic}
  \end{small}  
  \end{block}

\end{frame}

\begin{frame}
  \begin{block}{Pseudocode for the \texttt{MaxCrossing} function}
  \begin{tiny}
  \begin{algorithmic}
    \Procedure{MaxCrossing}{A}
       \State $Mid \gets \lfloor A.length / 2 \rfloor$
       \State

       \State $Sum \gets 0$
       \State $LeftSum \gets -\infty$
       \State $LeftIdx \gets Mid$

       \State
       \For{$I \gets Mid \ldots 1$}
         \State $Sum \gets Sum + A[I]$
         \If{$Sum > LeftSum$}
           \State $LeftSum \gets Sum$
           \State $LeftIdx \gets I$
         \EndIf
       \EndFor

       \State
       \State $Sum \gets 0$
       \State $RightSum \gets -\infty$
       \State $RightIdx \gets (Mid + 1)$

       \State
       
       \For{$J \gets (Mid + 1) \ldots A.length$}
         \State $Sum \gets Sum + A[J]$
         \If{$Sum > RightSum$}
           \State $RightSum \gets Sum$
           \State $RightIdx \gets J$
         \EndIf
       \EndFor

       \State
       \State {\bf return} $(LeftIdx, RightIdx, LeftSum + RightSum)$
    \EndProcedure
  \end{algorithmic}
  \end{tiny}
  \end{block}
\end{frame}

\section{Analysis and Asymptotic Notation}

\begin{frame}
  \frametitle{Analysing Algorithms}

  \begin{block}{Goal}
    \begin{itemize}
     \item Predict the resources that an algorithm requires
       \begin{itemize}
         \item computational time \pause (most often)
         \item memory, communication bandwidth, computer hardware
       \end{itemize}
     \item \ldots often ignoring tedious details
     \end{itemize}  
  \end{block}
\end{frame}

\begin{frame}
    \begin{block}{ General idea}
      \begin{itemize}
       \item The computational time taken to run an algorithm (such as
       InsertionSort) depends on the size of the input (e.g., the size $n$
       of the array) and how nearly sorted the elements of the array already are.
       \begin{itemize}
        \item We might describe the running time as a function of the input size
        \item The input size depends of the problem (the size of an array, the number of
          vertices and edges of a graph, \ldots). 
       \end{itemize}
    \end{itemize}
  \end{block}  

\end{frame}

\begin{frame}
  The computational running time of an algorithm on a particular input is the
  number of primitive operations or steps executed. \pause We
  assume that a constant amount of time is necessary to
  execute each line of a pseudocode. \pause Due to repetitions, the
  same statement might be executed many times. 

  \begin{block}{Example: InsertionSort}
    \begin{itemize}
    \item For a given array $A[1..n]$, our goal is to
      {\bf estimate} the running time of the algorithm ($T(n)$). 
    \end{itemize}
  \end{block}
\end{frame}


\begin{frame}{InserctionSort}

  \begin{block}{Pseudocode for Insertion Sort}
\begin{scriptsize}
    \begin{algorithmic}
    \Procedure{InsertionSort}{A}
       \For{$j = 2, \dots, A.length$}            
         \State $key \gets A[j]$                 
         \State $i \gets j-1$                    
         \While{$i > 0\ and\ A[i] > key$}        
            \State $A[i+1] \gets A[i]$           
            \State $i \gets i - 1$               
         \EndWhile
         \State $A[i+1] = key$                   
       \EndFor
    \EndProcedure
    \end{algorithmic}
  \end{scriptsize}  
  \end{block}
\end{frame} 

\begin{frame}
  The computational running time of an algorithm on a particular input is the
  number of primitive operations or steps executed. \pause We
  assume that a constant amount of time is necessary to
  execute each line of a pseudocode. \pause Due to repetitions, the
  same statement might be executed many times. 

  \begin{block}{Example: InsertionSort}
    \begin{itemize}
    \item For a given array $A[1..n]$, our goal is to
      {\bf estimate} the running time of the algorithm ($T(n)$).

    \item But note that the cost of the {\bf while loop} is
      a function of the variable $j$ (that is, we can
      model the {\bf while loop} running as ($T(j)$)
    \end{itemize}
  \end{block}
\end{frame}



\begin{frame}{InserctionSort}

  \begin{block}{Pseudocode for Insertion Sort}
\begin{scriptsize}
    \begin{algorithmic}
    \Procedure{InsertionSort}{A}
       \For{$j = 2, \dots, A.length$}             \Comment{Cost $c_1$ runs $n$ times}
         \State $key \gets A[j]$                  \Comment{Cost $c_2$ runs $n-1$ times}
         \State $i \gets j-1$                     \Comment{Cost $c_3$ runs $n-1$ times}
         \While{$i > 0\ and\ A[i] > key$}         \Comment{Cost $c_4$ runs $\sum_{j=2}^{n}t_j$ times}
            \State $A[i+1] \gets A[i]$            \Comment{Cost $c_5$ runs $\sum_{j=2}^{n}(t_j - 1)$ times}
            \State $i \gets i - 1$                \Comment{Cost $c_6$ runs $\sum_{j=2}^{n}(t_j - 1)$ times}
         \EndWhile
         \State $A[i+1] = key$                    \Comment{Cost $c_7$ runs $n-1$ times}
       \EndFor
    \EndProcedure
    \end{algorithmic}
  \end{scriptsize}  
  \end{block}
\end{frame} 

\begin{frame}
  \begin{multline*}
    T(n)  =  c_1n + c_2(n-1) + c_3(n-1) + c_4\sum_{j=2}^{n}t_j + c_5\sum_{j=2}^{n}(t_j - 1) \\ + c_6\sum_{j=2}^{n}(t_j - 1) + c_7(n-1)
  \end{multline*}
\pause
  \begin{itemize}
   \item Best case: array is already sorted
   \item Worst case: the array is in reverse sorted order 
  \end{itemize}
  
\end{frame}

\begin{frame}{Best case: $T(j) = 1$}
  \begin{multline*}
    T(n)  =  c_1n + c_2(n-1) + c_3(n-1) + c_4(n-1) + c_7(n-1) = \\
             (c_1 + c_2 + c_3 + c_4 + c_7)n - (c2 + c3 + c4 + c7) 
  \end{multline*}

  \pause
  
  \begin{itemize}
    \item $T(n)$ is a linear function of $n$.
  \end{itemize}

\end{frame}

\begin{frame}{Worst Case: $T(j) = j$, for $j=2, 3, \ldots, n$}

  \begin{block}{Note that}
    \begin{itemize}

      \item $\sum_{j=2}^{n}j = \frac{n(n+1)}{2}-1$

      \item $\sum_{j=2}^{n}(j-1) = \frac{n(n-1)}{2}$
      
    \end{itemize}
  \end{block}

\end{frame}
\end{document}


