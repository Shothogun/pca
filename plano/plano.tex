\documentclass[11pt]{article}

\usepackage{xcolor}
\usepackage{enumerate}

\title{Projeto e Complexidade de Algoritmos \\ (2022.2)}

\author{Rodrigo Bonif\'{a}cio}

\date{Outubro de 2022}

\newcommand{\eng}[1]{\emph{#1}}
\newcommand{\hl}[1]{\emph{{\color{blue}#1}}}
\begin{document}
\maketitle

\section{Objetivos}

Esta disciplina busca propiciar aos alunos conhecimentos
s\'{o}lidos em projeto e an\'{a}lise de algoritmos e uma
introdu\c c\~{a}o \`{a} Teoria da Complexidade Computacional.
Ao t\'{e}rmino da disciplina, o aluno deve ter um maior
dom\'{i}nio sobre (a) como projetar, compreender e
analisar algoritmos voltados para a resolu\c c\~{a}o
de diferentes tipos de problemas e (b) entender
os limites da computa\c c\~{a}o---incluindo uma perspectiva hist\'{o}rica
sobre o tema. Tamb\'{e}m se espera que essa
disciplina auxilie os alunos na compreens\~{a}o
de textos avan\c cados em computa\c c\~{a}o.

\section{Conte\'{u}do Program\'{a}tico}

\begin{enumerate}[(M1)]
\item Introdu\c c\~{a}o: Algoritmos de \hl{ordena\c c\~{a}o} (\eng{insertion sort},
  \eng{merge sort}), estrat\'{e}gia de resolu\c c\~{a}o
  baseada em \hl{divis\~{a}o-e-conquista} (\eng{merge sort}, \eng{maximum subarray}),
  como estudar algoritmos.

\item Complexidade de algoritmos, \hl{an\'{a}lise assint\'{o}tica} e \hl{rela\c c\~{o}es de recorr\^{e}ncia}.
  Prova de corretude de algoritmos.

\item A estrat\'{e}gia de resolu\c c\~{a}o baseada em \hl{program\c c\~{a}o din\^{a}mia} e a
  estrat\'{e}gia \hl{gananciosa} para resolu\c c\~{a}o de problemas computacionais. 

\item Algoritmos em \hl{grafos} (\eng{BFS}, \eng{DFS}, \eng{Topological Sorting}, \eng{Shortest Paths},
  \eng{Minimum Spanning Trees}).

\item Algoritmos \hl{ca\'{o}ticos} para \hl{otimiza\c c\~{a}o de programas} (incluindo \eng{control-flow graphs},
  \eng{reaching definitions}, \eng{very busy expressions}).

\item Algoritmos para \hl{aprendizagem de m\'{a}quina} e \hl{clusteriza\c c\~{a}o de dados}
  (dependendo da velocidade da turma).   

\item \hl {Complexidade Computacional} (Linguagens Regulares, Linguagens Livre de Contexto,
 A Tese de Church-Turing, Classes de Problemas: P, NP, NP-Completo)   
  
\end{enumerate}

\section{Avalia\c c\~{a}o}

Est\~{a}o previstas tr\^{e}s avalia\c c\~{o}es ao longo do per\'{i}odo.

\begin{enumerate}[P1]

\item Semin\'{a}rio em grupo de tr\^{e}s pessoas com a apresenta\c c\~{a}o
  de um algoritmo. Essa apresenta\c c\~{a}o deve descrever o algoritmo,
  a complexidade do algoritmo e a prova de corretude. Uma implementa\c c\~{a}o do
  algoritmo deve ser disponibilizada pelos alunos.

\item Implementa\c c\~{a}o de um algoritmo de an\'{a}lise est\'{a}tica
  de programas voltado para otimiza\c c\~{a}o (algoritmo de \emph{data-flow
  analysis}) ou de um algoritmo voltado para aprendizagem de m\'{a}quina.
  Os requisitos de implementa\c c\~{a}o devem ser discutidos com o professor.

\item Prova sobre o conte\'{u}do apresentando durante a disciplina.   
\end{enumerate}

A m\'{e}dia final ser\'{a} computada como: $MF = \frac{(P1 + P2 + P3)}{3}$


\section{Bibliografia}

\begin{itemize}

\item Thomas H. Cormen, Charles E. Leiserson, Ronald L. Rivest, and Clifford Stein. 2009. {\bf Introduction to Algorithms}, Fourth Edition (4th. ed.). The MIT Press.

\item Michael Sipser. 2012. {\bf Introduction to the Theory of Computation} (3rd. ed.). Cengage Learning.

\item Flemming Nielson, Hanne R. Nielson, and Chris Hankin. 2010. {\bf Principles of Program Analysis}. Springer Publishing Company, Incorporated.

\end{itemize}

\end{document}
